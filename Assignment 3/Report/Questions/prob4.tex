\documentclass[11pt,oneside,a4paper]{article}
\usepackage[utf8]{inputenc}
\date{}
\usepackage[linesnumbered,ruled,vlined]{algorithm2e}
\SetKwInput{KwInput}{Input}                % Set the Input
\SetKwInput{KwOutput}{Output}              % set the Output<z
\usepackage{blindtext}
\usepackage{changepage}


\usepackage[formats]{listings}  %% Code listing%% Code format

%\usemintedstyle{friendly}
\usepackage{amsmath}
\usepackage{color}
\usepackage{tcolorbox}
\usepackage{sectsty}
\usepackage{stmaryrd}
\usepackage{gensymb}
\usepackage{wasysym}
\usepackage{amsfonts}
\usepackage{xcolor}
\usepackage{graphicx}
\usepackage{stmaryrd}
\usepackage{mathtools}
\usepackage{amsthm}
\usepackage{caption}
\usepackage{subcaption}
\usepackage[margin=1.2in]{geometry}
%__________WIDE HAT_________
\usepackage{scalerel,stackengine}
\stackMath
\newcommand\reallywidehat[1]{%
\savestack{\tmpbox}{\stretchto{%
  \scaleto{%
    \scalerel*[\widthof{\ensuremath{#1}}]{\kern-.6pt\bigwedge\kern-.6pt}%
    {\rule[-\textheight/2]{1ex}{\textheight}}%WIDTH-LIMITED BIG WEDGE
  }{\textheight}%
}{0.5ex}}%
\stackon[1pt]{#1}{\tmpbox}%
}
\parskip 1ex
%______________________________
\usepackage{hyperref}
\hypersetup{
    colorlinks=true,
    urlcolor=blue
}
\setlength{\parindent}{0in}
\theoremstyle{definition}
\newtheorem{definition}{Definition}[section]

\theoremstyle{remark}
\newtheorem*{remark}{Remark}
\usepackage{amsmath}
\usepackage[T1]{fontenc}
\usepackage{float}
\usepackage[british,UKenglish,swedish,USenglish,english,american]{babel}
\usepackage[T1]{fontenc}
\usepackage{fancyhdr}
\usepackage{natbib}
%\pagestyle{fancy}
\begin{document}
\renewcommand{\bibname}{References}
\hypersetup{citecolor=black}
\begin{titlepage}\centering
\vspace*{\fill}
\Huge Assignment 2\\
\vspace*{10mm}
\large Includes methods of Directed Graphical Models, Dynamic programming, Variational Inference and Expectation Maximisation \\
\vspace*{\fill}
\large \textsc{DD2434 Advanced Machine Learning} \\
\textsc{Filip Bergentoft, bergento@kth.se} \\
\end{titlepage}

\newpage

Jag har gjort följande faktorisering

\begin{align*}
  p(A,Z,\Omega|G) = p(A,Z|\Omega, G)p(\Omega)
\end{align*}
Eftersom vår variational distribution enbart beror på $Z$ räcker det med att utveckla första termen i RHS. Låt $n^k_i(G_d)$ beskriva antalet infekterade arbetare $k$ träffade under dag $d$ och $\gamma$ beskriva hur många dagar en arbetare varit sjuk. Då får vi att

\begin{align*}
  p(A,Z|\Omega, G) & = \prod_{d,k}p(A_d^k|Z_d^k, \sigma) \prod_{d,k} p(Z_{d+1}^k|Z_d, G_d, \iota, \alpha) \\
  & = \prod_{d,k} \prod_{s,t,g,\gamma} p(Z_{d+1}^k = t | Z_d^k = (s,\gamma), n^k_i(G_d) = g, \iota, \alpha)^{I\big\{Z_{d+1}^k = t, Z_d^k = (s,\gamma), n^k_i(G_d) = g \big\}} \\
  & \prod_{d,k} \prod_{s,l} p(A_d^k = l| Z_d^k = s, \sigma)^{I\big\{A_d^l = l, Z_d^k = s \big\}}
\end{align*}

Verkar detta som en rimlig uppställning? Jag är en aning osäker på användandet av $n^k_i(G_d)$ men ser inte hur man kan göra på ett annat sätt givet de svar Jens gav Niko Palić i det öppna forumet. Skulle du ha möjlighet att indikera om detta är en god väg att gå eller om jag borde gå i en annan riktning?

\end{document}
