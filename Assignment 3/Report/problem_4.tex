\section*{3.4 Easier VI for Covid-19}
\begin{tcolorbox}
\textbf{Problem formulation:} We have a workplace with $K$ workers, $w_{1}, \cdots, w_{K},$ where we monitor Covid-19. Any day $d$ each worker $w_{k}$ is either non-infected, infected, or has antibodies, i.e., there is a latent variable $Z_{d}^{k}$ with a value in $\{n, i, a\},$ with the obvious interpretation. A non-infected individual becomes with probability $\iota$ infected the day after the individual has had contact with an infected individual (and though only one such contact may occur with any single infected individual during a day, an uninfected may have contact with several infected during a day). An individual that becomes infected on day $d$ is aware of the infection, and will on day $d+9$ get antibodies with probability $\alpha .$ Otherwise, the individual remains/returns to the non-infected state. An infected individual stays at home with probability $\sigma$ and is otherwise present at the workplace. We have access to a contact graph $G_{d}$ and an absence table $A_{d}$ for each day $d \in[D], A_{d}^{k}=1$ if worker $k$ is home on day $d$ and otherwise $0 .$ Consider $G=G_{d}$ as given so the joint is
$$
p(A, Z, \Omega \mid G)
$$
where $\Omega=(\iota, \alpha, \sigma) .$ There are beta priors on Bernoulli parameters $\iota, \alpha,$ and $\sigma .$ No other reasons than Covid-19 makes any worker stay at home. On day one $w_{1}$ is infected and all other workers are non-infected. Let $Z^{k}=Z_{1}^{k}, . . Z_{D}^{k}$ and $Z=Z^{1}, \ldots Z^{K} .$ Design a VI algorithm for approximating the posterior probability over $Z$ and use the VI distribution
$$
q(Z)=\prod_{d, k} q\left(Z_{d}^{k}\right)
$$
\end{tcolorbox}
During this problem I had a discussion with Ludvig Doeser regarding the problem formulation of the transition densities which were affected by how many infected workers one had met. This problem formulation was however simplified later on, although I after some time managed (I hope) to solve the harder case. 

\subsection*{Working the joint distribution}

We start off by simplifying the joint distribution keeping in mind that we are only interested in a variational approximation of the distribution for $Z$, yielding the complete likelihood.
\begin{align}
p(A,Z, \Omega | G) & = p(A,Z|\Omega, G)p(\Omega) \propto p(A,Z|\Omega, G) \nonumber
\end{align}
The complete likelihood can then be expanded to a product of emission and transmission probabilities.
\begin{align}
  p(A,Z|\Omega, G) = \prod_{d,k} p(A_d^k|Z_d^k, \sigma) p(Z_{d+1}^k|Z_d, G_d, \iota, \alpha)
  \label{complete_likelihood}
\end{align}
In order to enable us to keep track of when a worker should go from the state of infected to either the state of non-infected or the state of antibodies we expand the latent state to include the number of days a worker has been infected.
\begin{align}
  Z_d^k = \{s, \gamma\}, \; s \in \{n,i,a\}, \; \gamma \in \{0,1,...,8\}
\end{align}
\textit{Note:} in order to simplify calculations, $\gamma$ will only be expressed on the conditioning side of the transition probability.
We will in addition introduce a counter $n_d^k = g, \; g \in [K]$ that given $Z_d^{-k}$ and $G_d$ tells how many infected workers worker $k$ has met during day $d$. This is useful when computing the transition probabilities. \\

The \textit{emission probabilities} can then be expressed as
\begin{align}
  p(A_d^k|Z_d^k, \sigma) & = \prod_{s,l} {\underbrace{p(A_d^k=l|Z_d^k=s, \sigma)}_\text{$E_{sl}$}} ^{I\{A_d^k=l, Z_d^k=s\}} \nonumber \\
  & = \prod_{s,l}  E_{sl}^{I\{A_d^k=l, Z_d^k=s\}}
\end{align}

We will later make use of the following \textbf{emission matrix} which describes the values for
$$E_{sl} = p(A_d^k=l|Z_d^k=s, \sigma)$$

\begin{table}[H]
  \begin{center}
      \begin{tabular}{| g | c | c |}
      \hline
      \rowcolor{LightCyan}
      $ $ & $l=0$ & $l = 1$ \\ \hline
      $s=n$ & $1$ & $0$ \\ \hline
      $s=i$ & $1-\sigma$ & $\sigma$ \\ \hline
      $s=a$  & $1$ & $0$ \\ \hline
      \end{tabular}
  \end{center}
\caption{Emission probabilities for $E_{sl}$ in white background}
\label{emission_matrix}
\end{table}

The \textit{transition probabilities} can then be expressed as
\begin{align}
  p(Z_{d+1}^k|Z_d, G_d, \iota, \alpha) & = \prod_{s,t,g,\gamma}
  {\underbrace{p(Z_{d+1}^k = t|Z_d^k = \{s,\gamma\}, n_d^k = g, \iota, \alpha)}_\text{$T_{stg\gamma}$}} ^{I\{Z_{d+1}^k = t, Z_d^k = \{s,\gamma\}, n_d^k = g\}} \nonumber \\
  & = \prod_{s,t,g,\gamma} T_{stg\gamma}^{I\{Z_{d+1}^k = t, Z_d^k = \{s,\gamma\}, n_d^k = g\}}
\end{align}
Substituting the emission and transition probabilities into equation \eqref{complete_likelihood} yields
\begin{align}
  p(A,Z|\Omega, G) = \bigg( \prod_{d,k} \prod_{s,l} E_{sl}^{I\{A_d^k=l, Z_d^k=s\}} \bigg) \bigg( \prod_{d,k} \prod_{s,t,g,\gamma} T_{stg\gamma}^{I\{Z_{d+1}^k = t, Z_d^k = \{s,\gamma\}, n_d^k = g\}} \bigg)
\end{align}

We will later make use of the following \textbf{transition matrix} which describes the values for
$$T_{stg\gamma} = p(Z_{d+1}^k = t|Z_d^k = \{s,\gamma\}, n_d^k = g, \iota, \alpha)$$


\begin{table}[H]
  \begin{center}
      \begin{tabular}{| g | g | g | c | c | c |}
      \hline
      \rowcolor{LightCyan}
      &  &  & $t = n$ & $t = i$& $t = a$ \\ \hline
      $s=a$ & $\forall \gamma$ & $\forall g$ & $0$ & $0$ & $1$ \\ \hline
      $s=i$ & $\gamma<8$ & $\forall g$ & $0$ & $1$ & $0$ \\ \hline
      $s=i$ & $\gamma=8$ & $\forall g$ & $1-\alpha$ & $0$ & $\alpha$ \\ \hline
      $s=n$ & $\forall\gamma$ & $g=g$ & $(1-\iota)^g$ & $1 - (1-\iota)^g$ & 0 \\ \hline
      \end{tabular}
  \end{center}
\caption{Transition probabilities for $T_{stg\gamma}$ in white background}
\label{transition_matrix}
\end{table}


\subsection*{Update equations}

Given the \textit{variational distribution} for $Z$
\begin{align}
  q(Z) = \prod_{d,k} q(Z_d^k)
\end{align}
we need to compute
\begin{align}
  \log q^*(Z_x^y) & \propto \underset{\{d,k\} \neq \{x,y\}}{\mathbb{E}} \bigg[ \log p(A,Z|\Omega, G) \bigg] \nonumber \\
  & = \underset{\{d,k\} \neq \{x,y\}}{\mathbb{E}} \bigg[\sum_{d,k}\sum_{s,l} I\{A_d^k=l, Z_d^k=s\} \log E_{sl} \bigg] \label{emission} \\
      & + \underset{\{d,k\} \neq \{x,y\}}{\mathbb{E}} \bigg[ \sum_{d,k}\sum_{s,t,g,\gamma} I\{Z_{d+1}^k = t, Z_d^k = \{s,\gamma\}, n_d^k = g\} \log T_{stg\gamma} \bigg] \label{transmission}
\end{align}

We start by working with the emission term in equation \eqref{emission}. Keeping in mind that we are only interested in $Z_x^y$ and that we are only taking the expectation with respect to all $d, k$ except $\{d,k\} \neq \{x,y\}$. $Z_x^y$ can thus be seen as a constant with respect to the expectation and can thus be moved out of it. This yields that

\begin{align}
  \underset{\{d,k\} \neq \{x,y\}}{\mathbb{E}} \bigg[\sum_{d,k}\sum_{s,l} I\{A_d^k=l, Z_d^k=s\} \log E_{sl} \bigg] \propto
  \sum_{s,l} \log E_{sl} I\{A_x^y=l, Z_x^y=s\}
  \label{expanded_emission}
\end{align}

Using the same mindset we can simplify the transmission term in equation \eqref{transmission} as follows

\begin{align}
  & \underset{\{d,k\} \neq \{x,y\}}{\mathbb{E}} \bigg[ \sum_{d,k}\sum_{s,t,g,\gamma} I\{Z_{d+1}^k = t, Z_d^k = \{s,\gamma\}, n_d^k = g\} \log T_{stg\gamma} \bigg] \nonumber\\
  & \propto \sum_{s,t,g,\gamma} \log T_{stg\gamma} \underset{\{d,k\} \neq \{x,y\}}{\mathbb{E}} \bigg[
  I\{Z_{x}^y = t\} I\{Z_{x-1}^y = \{s,\gamma\}\} I\{n_{x-1}^y = g\} + \nonumber \\
  & \qquad\qquad + I\{Z_{x+1}^y = t\} I\{ Z_x^y = \{s,\gamma\}\} I\{n_x^y = g\}\bigg] \nonumber \\
  & = \sum_{s,t,g,\gamma} \log T_{stg\gamma} \bigg( I\{Z_x^y = t\}P(Z_{x-1}^y = \{s,\gamma\})P(n_{x-1}^y = g) + \nonumber \\
  & \qquad\qquad + I\{Z_x^y = \{s,\gamma\}\}P(Z_{x+1}^y = t)P(n_x^y = g) \bigg)
  \label{expanded_transmission}
\end{align}

Now, in equation \eqref{expanded_transmission} $P(Z_{x+1}^y)$ and $P(Z_{x-1}^y)$ can be computed using the values $q(Z_{x+1}^y)$ and $q(Z_{x-1}^y)$ for the previous VI iteration. However we need to compute the probability $P(n_x^y = g)$ which corresponds to the probability of worker $y$ having met $g$ infected individuals during day $x$, given the contact graph $G_x$. \\

$P(n_x^y = g)$ can be computed by using a function such as the "nchoosek" function in Matlab. Letting $v$ be the column in the contact graph $G_x$ representing the workers that worker $y$ has met during day $x$, $C = \text{nchoosek}(v, g)$ returns a matrix $C$ containing all possible combinations of the elements of vector $v$ taken $g$ at a time. Each row in $C$ then represents a possible permutation of $g$ infected workers, where $c_{ij}$ represents the index of a worker. The probability is then given by

\begin{equation}
  P(n_x^y = g) = \sum_i \prod_j^g q(Z_x^{c_{ij}})
  \label{contact_prob}
\end{equation}

%qqq: Must add how to compute $P(n_x^y = g)$ here !!!!!!! - It will be computed by using the update probabilities.


Substituting \eqref{expanded_emission} and \eqref{expanded_transmission} into \eqref{emission} and \eqref{transmission} respectively yields
\begin{align}
  \log q^*(Z_x^y) & \propto \sum_{s,l} \log E_{sl} I\{A_x^y=l, Z_x^y=s\} + \nonumber\\
  & + \sum_{s,t,g,\gamma} \log T_{stg\gamma} \bigg( I\{Z_x^y = t\}P(Z_{x-1}^y = \{s,\gamma\})P(n_{x-1}^y = g) + \nonumber \\
  & + I\{Z_x^y = \{s,\gamma\}\}P(Z_{x+1}^y = t)P(n_x^y = g) \bigg)
  \label{final_eq}
\end{align}

\subsection*{Estimating parameters}
Now we have everything we need to start the algorithm except any deterministic values for the parameters in $\Omega$. These have to be estimated in order for us to have any values for $\log(E_{sl})$ and $\log(T_{stg\gamma})$. Given that we have beta priors on all Bernoulli parameters $\iota, \alpha, \sigma$ we can get the values for these by taking the expectation of $\log(E_{sl})$ and $\log(T_{stg\gamma})$ with respect to the beta distribution. These have closed form solutions and can be computed using wolframalpha for example. For example if $\sigma \sim \text{Beta}(a,b)$ then
$$\mathbb{E}[\log(\sigma)] = \psi(a) - \psi(a + b)$$
where $\psi$ is the digamma distribution. The remaining probabilities in the emission and transmission matrices can be computed in the same analog manner.


\subsection*{The algorithm}
The algorithm can now be run in the following manner using the following relationship

\begin{align}
  \log q^{i+1}(Z_x^y) & \propto \sum_{s,l} \log E_{sl} I\{A_x^y=l, Z_x^y=s\} + \\
  & + \sum_{s,t,g,\gamma} \log T_{stg\gamma} \bigg( I\{Z_x^y = t\}q^{i}(Z_{x-1}^y = \{s,\gamma\})P(n_{x-1}^y = g) + \nonumber \\
  & + I\{Z_x^y = \{s,\gamma\}\}q^{i}(Z_{x+1}^y = t)P(n_x^y = g) \bigg)
\end{align}

\textbf{Algorithm:}
\begin{enumerate}
  \item $i = 0$
  \item Initiate $q^{i}(Z_x^y) \; \forall x,y$ as a proper density
  \item Run until convergence \begin{itemize}
    \item For all $x,y,g$ compute $P(n_x^y = g)$ with probabilities $q^{i}(Z)$ using equation \eqref{contact_prob}
    \item For all $x,y$ compute $\log q^{i+1}(Z_x^y)$ using equation \eqref{final_eq}, expectation values for $\log(E_{sl})$ and $\log(T_{stg\gamma})$ computed in the previous section and probabilities $P(n_x^y = g)$ computed on the previous line
    \item For all $x,y$ set $q^{i+1}(Z_x^y) = \frac{\exp\{\log q^{i+1}(Z_x^y)\}}{\sum_{x,y}\exp{\log q^{i+1}(Z_x^y)}}$
    \item Set $i = i + 1$

  \end{itemize}
\end{enumerate}
